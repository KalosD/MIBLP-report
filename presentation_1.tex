\documentclass{beamer}

%%%%% ===== 设置主题 *****
\usetheme{Berlin}
% 可供选择的主题参见 beameruserguide.pdf
% 无导航条的主题: Bergen, Boadilla, Madrid, CambridgeUS,
%                 GoettingenAnnArbor,Pittsburgh, Rochester;
% 有树形导航条的主题: Antibes, JuanLesPins, Montpellier;
% 有目录竖条的主题: Berkeley, PaloAlto, Goettingen, Marburg, Hannover;
% 有圆点导航条的主题: Berlin, Ilmenau, Dresden, Darmstadt, Frankfurt, Singapore, Szeged;
% 有节与小节导航条的主题: Copenhagen, Luebeck, Warsaw

\useinnertheme{circles}
\useoutertheme{infolines}
\usefonttheme[onlymath]{serif}
\setbeamertemplate{navigation symbols}{} % remove the navigation
\setbeamersize{text margin left=0.8cm, text margin right=0.8cm}
\setbeamerfont{frametitle}{size=\large}
\setbeamerfont{footline}{family=\ttfamily}
%%%%% ===== 宏包 *****
\usepackage{amsmath,amssymb,amsfonts}
\usepackage{graphicx,xcolor}
\graphicspath{{figure/}}
\usepackage{hyperref}
\hypersetup{breaklinks=true}
\usepackage{bm}
\usepackage{amsthm}
\usepackage{ulem}
% \usepackage{algorithmicx,algorithm}
\usepackage[linesnumbered,ruled,vlined]{algorithm2e}
\usepackage{fontspec}
\usepackage{xeCJK}
%\usepackage{biblatex}
%\bibliographystyle{plain}
\usepackage[backend=bibtex,style=numeric,sorting=none]{biblatex}
%\addbibresource{example.bib} %BibTeX数据文件及位置
\addbibresource{bibtex3.bib}
\setbeamerfont{footnote}{size=\tiny}
%\usepackage[round,sort&compress]{natbib}
\renewcommand{\baselinestretch}{1.1}

%%%%% ===== 自定义命令 *****
\newcommand{\myem}[1]{\textcolor{blue}{#1}}
\renewcommand{\today}{\number\year 年\number\month 月\number\day 日}
%\newtheorem{theorem}{Theorem}

\begin{document}
\title[新能源与储能容量配置优化研究]%
{新能源与储能容量配置优化研究}


\author[杜洪博]%
{报告人: 杜洪博\\
导\quad 师: 寇彩霞\rule[0pt]{0pt}{20pt}\\}
%  \textcolor{red}{\texttt{xxxx@xxxx.xxx}}}

\institute[BUPT]{\textcolor[rgb]{0.0,0.0,0.10}%
{\small\ttfamily 北京邮电大学\ 理学院\\[10pt]}}

\date{\today}

% ===== title page ====================================
\begin{frame}[plain]
	\titlepage
\end{frame}

\begin{frame}
	\frametitle{目录}
	\tableofcontents[hideallsubsections] %[pausesections]
\end{frame}

\AtBeginSection[] % Do nothing for \section*
{ \begin{frame}<beamer> %\frametitle{Outline}
		\tableofcontents[currentsection,hideallsubsections]%,currentsubsection]
	\end{frame}
}

%===== Main part start here ==========================
%===== 开题报告 =======================================




\section{问题背景与研究意义}

\begin{frame}
	\frametitle{问题背景} 
	
\end{frame}

\begin{frame}
	\frametitle{研究意义} 

\end{frame}

\section{研究现状}

\begin{frame}
	\frametitle{研究现状} 
    
\end{frame}

\begin{frame}
	\frametitle{研究现状} 

\end{frame}

\section{研究内容与方法}

\begin{frame}
		\frametitle{研究内容} 
	
\end{frame}

\begin{frame}
	\frametitle{研究方法} 

\end{frame}

\section{计划与预期成果}
\begin{frame}
	\frametitle{计划与预期成果} 
			    
\end{frame}



% ====== End =========================================
\begin{frame}
\vspace{1em}
\centering
\textcolor{blue}{\LARGE Thanks for your attention!}

\end{frame}

\end{document}
